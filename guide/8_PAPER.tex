\chapter{Research projects}%
	\label{ch:paper}

\newthought{The research project} is the graded coursework component that you work on with a student partner throughout the entire semester. Projects start on Week~2 and end on ``Week~13'', one week after the final course session, when you submit your final paper. Twice during the semester, you will submit drafts of that paper, based on templates. You will also submit some code, in the form of a Stata do-file, to replicate your analysis.

% 1. register your project
% 2. download the template
% 3. 

% Project management

 % % % % % % % % % % %

%%% -- old

\section*{First draft: data preparation}

% The course is built around your elaboration of a small-scale research project. Because the course is introductory by nature, several limits apply:
% 
% −	You are required to use pre-existing data, instead of assembling your own data and building your own dataset, which is a much longer process that require ad-ditional skills in data manipulation.
% −	You are required to use cross-sectional data, because time series, panel and longitudinal data require more complex analytical procedures that are not cov-ered in introductory courses.
% −	You are required to use continuous data, because discrete variables also use different techniques not covered at length in the course. This applies principally to your choice of dependent variable.
% 
% These requirements and their terminology are covered in Section 5. For now, just re-member this basic principle: your research will be based on one dependent variable, sometimes called the `response'' variable, and you will try to explain this variable by predicting the different values that it takes in your sample (dataset) by using several independent variables, sometimes called the `explanatory'' variables, or `predictors'', or `regressors'' in technical papers.
% 
% Because your research is a personal project, you might bend the above rules to some extent if you quickly show the instructors that you can handle additional work with da-ta management. The following advice might then apply: 
% 
% −	If you are assembling your data by merging data from several datasets, you will be using the merge command in Stata (Section 5.2). You might also choose to use Microsoft Excel for quicker data manipulation. Do not assemble data if you do not already have some experience in that domain.
% −	If you are converting your data, refer to the course and online documentation on how to import CSV data into Stata using the insheet command, or how to convert file formats like SAV files for SPSS. Always perform extensive checks to make sure that your data were properly converted into a readable, valid file.
% −	If you are interested in temporal comparison, such as economic performance before and after EU accession, you can compute a variable that will capture, for example, the change in average disposable income over ten years. Stick with a single variable, and ask for advice in class before proceeding.
% −	If you have selected nominal data as your dependent variable, such as religious denomination, then something went wrong in your research design—unless you know about multinomial logit, in which case you should skip this class. Please identify a different variable that is either continuous or `pseudo-continuous''—i.e. a categorical variable with an ordinal (or better, interval) scale, such as edu-cational attainment.

	\subsection{Dataset selection}

	\subsection{Variable selection}

% /* Notes on selecting variables for analysis:
% 
%  - The first and most important choice is the dependent variable (DV), which has
%    to be a continuous variable, that is, a variable that takes many intervalled
%    values on a given scale. You must fully understand that aspect of the DV.
% 
%  - Some limitation in continuity is acceptable. The example provided here is a 
%    borderline case, as the DV is measured on a 5-point scale that makes it more 
%    categorical than continuous.
%    
%  - The reliability of your final analysis will be either more limited or more
%    complex if you choose a dependent variable that is less continuous than it is
%    categorical. There will be one late session on logistic regression for these.
% 
%  - Practical consequences: use either country-level data from the Quality of
%    Government dataset, or use variables from the ESS, GSS and WVS surveys that
%    were measured on 7-pt or 10-pt scales, or something even higher than that.
% 
%  - You can violate that recommendation in two cases:
%  
%    (1) You are interested in having a particular categorical variable as your
%        dependent variable, and you are willing to go one extra mile in the late
%        stage of this course by learning additional modelling.
%    
%    (2) You have found a categorical variable that is pseudo-continuous, i.e.
%        that follows some approximation of the normal distribution. We will cover
%        the normal distribution next week, so stay tuned.
% 
%    In either case, you will have to proceed cautiously. The main skill here is
%    the ability to understand the structure of your data, as you would in other
%    research situations that rely on different forms of research material.
% 

%  - The next step is to select independent variables (IVs). Some of them will 
%    work as controls: you want to make sure that you understand how your DV
%    relates to persistent, structural factors that are hard to affect altogether.
% 
%  - Some other independent variables will operate as DV predictors: you want to
%    test the idea that your DV is determined by these IVs, by showing that the
%    values taken by these IVs can predict those taken by the DV.
% 
%  - Finally, some independent variables might be more covariates than predictors,
%    i.e. variables that you predict to be strongly associated to the DV and to
%    vary in related directions (as with "the higher..., the lower...").
% 
%  - If you are going to work with survey (individual-level) data, include common
%    socio-demographics like age, sex, education, household composition, e.g.
%    marital status and number of children (and possibly household income).
%    
%  - If you are going to work on country-level data, include at least one variable
%    for economic development (the most common measure is GDP per capita) and one
%    measure of political organization (e.g. democratic v. dictatorial regime). 
%  
%  - Quickly check a few academic references on your topic to see what variables
%    might be relevant. The course contains a few example papers, and other ones
%    are easy to locate using Google Scholar with precise keywords. */

	% missing values, recodes, summary stats
