%
%
%
\section{Descriptive statistics}
  %
  %
  \label{sec:summary}%
  %
  %
  \index{Statistics!Summary statistics|see{Descriptive statistics}}%
  %
  %

Descriptive statistics are ways to summarize your data. Measures of central tendency are used to reduce a sequence of values to one parameter, such as the average number of children per household. Measures of dispersion indicate how the values spread around this parameter. The combination of both measures tell you about the distribution of a variable.%

	%
	% 2.1.1
	%
	\subsection{Measures of central tendency}%
  	\index{Descriptive statistics!Central tendency}%
    \index{Descriptive statistics!Central tendency!Means}\index{Means}
  	\index{Central tendency|see{Descriptive statistics}}%
    

    %
    %
    %
    \paragraph{Median}%
    	\index{Descriptive statistics!Central tendency!Median}%
    	\index{Median}%
    %
    Arithmetic mean values are very sensitive to outlier observations, but this issue can be solved by using the \textbf{median} instead, which is the \hlred{``middle'' value of the distribution}: 50\% of the values fall below the median, 50\% of the values fall above the median. Unlike the mean, the median is a robust measurement because you need to alter a large fraction of the distribution to displace its value.

    \paragraph{Mode}%
    \index{Descriptive statistics!Central tendency!Mode}\index{Mode}

  %
  %
  %
  \paragraph{Proportions}%
  	\index{Descriptive statistics!Proportions|see{Frequencies}}%
  	\index{Descriptive statistics!Frequencies}%
  	\index{Proportions|see{Frequencies}}%
  	\index{Frequencies}

  When the values taken by a variable are not continuous, measures of central tendency and dispersion are not meaningful parameters of their distribution. Categorical variables, for which numerical values code categories, are instead addressed by looking at their \hlred{frequencies}:%

  \begin{itemize}
  	\item The \textbf{absolute frequency} of a category is the number of observations that it qualifies. If your value represents a group, as when you code ethnicity with numbers, then the absolute frequency is the number of observations that fall into that group. The group with most observations is the modal category, \ie the majority group.%

  	\item The \textbf{relative frequency} of a category is its absolute frequency divided by the total number of observations in the sample. This fraction is routinely provided as a proportion $0 \leq p \leq 1$, or as its percentage $100 \cdot p$. For example, women usually represent slightly above $\frac{1}{2}=0.5=50\%$ of the population.%
  \end{itemize}

	%
	% 2.1.2
	%
	\subsection{Dispersion around the mean}%
	\index{Descriptive statistics!Dispersion}%
	\index{Dispersion}%


	%
	% 2.1.3
	%
	\subsection{Exporting descriptive tables}
    \index{Descriptive statistics!Table formatting}%

  \label{cmd:stab}%
  The \cmd{stab} command is a wrapper for some of the export functions in the \cmd{estout} package It is part of the course setup that you ran at p.~\pageref{sec:course-setup}, lets you export plain text tab-separated tables of summary statistics. These tables are convenient to open in spreadsheet editors like Microsoft Excel or OpenOffice Calc, and can be pasted into Google Documents.%

  There are clunkier ways to export summary statistics in more sophisticated ways with \cmd{tabout} and its supplementary command \opt{tabstatout}{tabout}, but you should be fine with \cmd{stab}.%
