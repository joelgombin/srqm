%
% 1.3
%
\section{Working with Stata}%
	%
	%
	\label{sec:stata}%
	\index{Stata!Introduction}%
	%
	%
	
		
% 0.3.1
%
\subsection{Learning statistical software}%

%
%
\newthought{You have probably never used Stata}, but you probably have some limited experience with number-crunching. For most users, this happens in a numerically inaccurate environment called a spreadsheet editor, in which you can freely delete or edit the data without letting others know. In that kind of environment, programming means writing functions into cells and then hoping for the cells to stay where they are, which they never do (Figure~\ref{fig:dilbert-spreadsheets}).%
	%
	\footnote{Read some horror stories compiled by the European Spreadsheet %
	Risks Interest Group (EuSpRIG): %
	\url{http://www.eusprig.org/horror-stories.htm}} %
	%
	A vast majority of jobs done in spreadsheet environments will not even explore that territory and will instead safely focus on showing unreadable `3D' pie charts with no meaningful third dimension.%
	%
	%

	\begin{figure}
		\includegraphics{dilbert-excel}
	  \caption{\href{http://dilbert.com/strips/comic/2007-08-08/}{Dilbert on spreadsheets}, by Scott Adams.}%
	  \label{fig:dilbert-spreadsheets}%
	\end{figure}

Your use of Stata in this course will remind you of a few things that you might have already learnt from using spreadsheet software. This will include using keyboard shortcuts a lot to move around the interface and to do things quicker, especially with respect to text manipulation. It will also include using filters and operators like \texttt{IF} or \texttt{MEAN} that you might know from functional programming in spreadsheets.%
	%
	%

Statistical programming, on the other end, is probably new to you. Its general principle works more or less like writing music for a single musician: you have to compose a sequence of instructions that enables anyone to play your stuff. The current de facto standard in statistical programming is R, a free and open source software with excellent graphics support that lets you crunch numbers like a statistician would want you to: neatly, if at the cost of intuitiveness.%
	%
	\footnote{For example, in R, the number `2' is not just a number (\ie a scalar), it is a vector of length 1 of numeric class.} %
	%
 	R users know that its sheer complexity and subtle idiosyncracies create a steeper learning curve for new users, and that there is ``something not fully healthy''%
	\footnote{%
		\url{https://stat.ethz.ch/pipermail/r-help/2007-August/139047.html}} %
		about it. Still, if you ever plan to enhance your statistical programming skills, R and a few other languages will become your weapon of choice. Ivaylo and myself also teach an exploratory course in R if you want to get a feel of it.%
	\footnote{\url{http://f.briatte.org/teaching/ida/}}%
%
%

\newthought{Stata is a compromise} between these two worlds that lets you work on a single spreadsheet of data through a set of predefined and user-contributed commands. It compares well to the competition: its emphasis on programming is more pronounced than what it is in SPSS, which is mostly used through point-and-click, and its language is more convenient than the one used by SAS, an expensive and historically dominant statistical software that remains the only solution capable of dealing with very large datasets. Stata is also less focused on a single approach to quantitative analysis than, say, Epi~Info for epidemiology or gretl for econometrics.%
	%
	%
	
Stata has its own limitations. Its graphics engine is not bad, but it is far from excellent: the graph syntax is tiresome, and it takes lines and lines of code to escape the default plots to produce better visualizations. Despite being rather open to user-contributed commands, Stata is a commercial product with a price tag and a closed proprietary format that requires conversion to import outside of Stata. Finally, and although that can be seen as a feature of the software, Stata can only work on a single dataset at a time, because it works like an accountant's book, one page after the other.%
	%
	%

% 0.3.2.
%
\index{Computers!Graphical User Interface (GUI)}%
\index{Computers!Command Line Interface (CLI)}%
\index{Stata!Command line interface}%
%
\subsection{Working from the command line} %
  \label{sec:cli}
%
% graphical user interface
% command line
% example: sysuse lifeexp
%
Stata can be used either through its Graphical User Interface (GUI), like most of the software that you use, or through a Command Line Interface (CLI), which presents itself as a terminal where the user types instructions to produce certain results. The latter approach is the programmer's way of doing things: it is much more versatile and forces you to write up your operations in the form of a script, which Stata calls a `do-file'. This file is a plain text file containing Stata code that, like a music sheet, makes it possible to understand and reproduce your analysis of the original data.%

We are going to explore Stata from the command line, which relies only on three windows: the Command line window, the Results window attached to it, and the do-files window. This minimal setup is shown below in Figure~\ref{fig:stata-ide}. We will not use any other element of the graphic user interface in this course, but feel free to explore Stata and learn about its other functionalities.%

\begin{figure}
	\includegraphics{stata-ide}%
	\caption{The Holy Trinity of Stata programming: the Command window (on top), the Results window (below it), and a do-file editor (in the background).}%
	\label{fig:stata-ide}%
\end{figure}

%% -- command line

% The Stata GUI can be used occasionally for routine operations that need not appear in your do-files. Keyboard shortcuts also save some time, as with ‘File > Open…’ (Ctrl-O in Stata for Windows) or ‘File > Change Working Directory…’ (Cmmd-Shift-J in Stata for Macintosh; this document uses ‘Cmmd’ to designate the ‘’, a. k. a. ‘Command’, ‘Cmd’ or ‘Apple’ modifier key).

The next sections explain how to run commands in Stata, and how to organize them into a do-file. A list of all commands used in the guide appear in the index, and a list of all commands used in the course appear on the course wiki.%
	\footnote{\url{https://github.com/briatte/srqm/wiki/code}}%

	\paragraph{Running commands}
  
	In Stata, focus on the Command window by clicking into it or by pressing \kbd{Cmd-1} (Mac) or\kbd{Ctrl-1} (Win). Type the following command and press \texttt{Enter} to run the command, which means to execute its code:%
		
	\begin{docspec}%
		\label{lifeexp}%
		sysuse lifeexp, clear
	\end{docspec}
		
	A Stata command is simply a line of code written on a distinct line of a do-file. If your command ran successfully, Stata will display its result:\\[1em]
		
		\includegraphics{stata-lifeexp-sysuse}\\[1em]

	Note that some commands produce `blank' output, which means that a command can be successfully entered and executed without printing any result. In this case, a simple \texttt{.} line dot will appear in the Results window, as to show that Stata encountered no problem while executing the command, and that it is ready to process another one.%
		%

  %
  %
	\paragraph{Fixing typos}%
			\index{Stata!Syntax}%

	Stata commands follow a strict syntax, and if you make a typo in a command, the software will return an error in red ink. In that case, you have to fix the issue by re-typing the command correctly. Try for example, to enter the following command:%
		
		\begin{docspec}
			summarizze popgrowth
		\end{docspec}%
    \marginnote{Examples with the \texttt{lifeexp} dataset continued from p.~\pageref{lifeexp}.}%
		
		The mistake is rather obvious here: the \cmd[su]{summarize} command should take only one `z':\\[1em]%
		
		\includegraphics{lifeexp-sum-error-zz}\\[1em]
		
		If you need, as in this example, to correct a mispelled command, or to re-run a command that you have used earlier on, you should \textbf{press \kbd{PageUp} on your keyboard} to recall the previous command that you typed.%
		%
		\footnote{On laptop keyboards, \kbd{PageUp} is usually replaced by \kbd{Fn-UpArrow}. Past commands can also be examined in the Review window.} %
		%
		This allows you to fix your mistake by pulling the last command very quickly and correct only the typo, rather than having to type it again.%

		Here's another common mistake. Try the following in Stata:%
		
		\begin{docspec}
			Summarize popgrowth
			summarize POPGROWTH
		\end{docspec}

		In response to these commands, you will get more or less informative error messages. The issue here is that Stata commands are case-sensitive: the capitalized \texttt{Summarize} command is different from \texttt{summarize} command in lowercase. By that virtue, there is no variable called \texttt{POPGROWTH} in uppercase in the \texttt{lifeexp} dataset, but there is one called \texttt{popgrowth} in lowercase:\\[1em]%

		\includegraphics{lifeexp-sum-error-caps}\\[1em]
		
		Stata is usually operated in lowercase, although some variable names might show up in uppercase in some datasets. In order to keep things as straightforward as possible, avoid using uppercase yourself when naming variables.%
		
    %
    %
		\paragraph{Abbreviations}%
			\index{Stata!Syntax}%

		Most Stata commands can be abbreviated for quicker use. If you run the \statacode{help summarize} command, the help window will tell you that the \texttt{\underline{su}mmarize} command can be abbreviated to \texttt{su}. Type the following example in Stata to see how the language abbreviates:%
    
  	\begin{docspec}
  		* With a command:\\%
  		summarize popgrowth\\%
  		su popgrowth\\[1em]%
  		%
  		* With help pages:\\%
  		help summarize\\%
  		h su%
  	\end{docspec}%
  			\marginnote{Examples with the \texttt{lifeexp} dataset continued from p.~\pageref{lifeexp}.}%

  Note that the bottom lines show you how to open help pages with the single letter \texttt{h}, which is handy when you are often brought to verify syntax and read examples from the documentation. This guide shows commands that can be abbreviated in both forms, as with \cmd[su]{summarize}, which is shown in the example:\\[1em]%

  	\includegraphics{lifeexp-summarize}\\[1em]

  Abbreviations exist for most commands and come in handy especially with commands such as \cmd[tab]{tabulate}, \cmd[d]{describe} or even \cmd[h]{help}. They also work for options like the \coab{d}{detail}{summarize} option for the \cmd[su]{summarize} command:\\[1em]%

  	\includegraphics{lifeexp-summarize-d}\\[1em]
  
		%
		%
		\paragraph{Using help files}%
			\index{Stata!Help files}

		If you cannot find the right syntax or option for a command, turn to the technical documentation for precisions and examples of any given command. Use the \cmd[h]{help} command to access it in Stata,%
    		\footnote{Stata help is also available online: \url{http://www.stata.com/help.cgi?help}.} %
        as in these examples:%

		\begin{docspec}
			* Help for the -summarize- command.\\
			help summarize\\[1em]
	
			* Help for the -lookfor- command.\\
			help lookfor
		\end{docspec}

    Getting to use the Stata help pages is a course objective in itself: even experienced Stata users use help pages on a daily basis. The details on each option and the final examples are often very useful in learning to use some commands efficiently.%

    \newthought{You can find additional help} for Stata in the series of manuals and handbooks published by Stata Press. Appropriate references for this course are the ``Getting Started'' manuals, which cover the same basic operations as this guide.%
      \footnote{\url{http://www.stata-press.com/}} %
      StataCorp also publishes the \emph{Stata Journal}, a blog and a video channel of Stata tutorials.%
      \footnote{\url{http://blog.stata.com/2012/09/26/stata-youtube-channel-announced/}}
  
    The course website features a list of additional online resources to learn Stata. One particularly interesting resource for beginners is the Stata video series by the LSE Methodology Institute.%
      \footnote{\url{http://www.youtube.com/user/MethodologyLSE/videos?query=stata}} %
      Stata users also share questions and answers on Statalist%
      \footnote{\url{http://stata.com/statalist/}} %
      and on Stackoverflow.%
      \footnote{\url{http://stackoverflow.com/questions/tagged/stata}}

%
%
\paragraph{Installing commands}%
	\index{Stata!Packages (additional commands)}%

Stata can install additional commands written by users with programming skills. New commands can be installed by downloading packages from the \SSC server with the \cmd{ssc install} command, which is used at a few points in this guide to install some of these packages. You can install the \cmd{fre} command right away by typing the following command:%
	
\begin{docspec}
	ssc install fre
\end{docspec}
	
Unless you are offline or already have installed the \cmd{fre} package, you should get a few result lines indicating where the package was installed on your hard drive:\\[1em]%
	
\includegraphics{ssc-install-fre}\\[1em]
	
Other handy user-contributed commands will be installed as part of the course setup that you will run in the next section.%

%
%
\subsection{Working from a do-file}%
	\label{sec:do-files}%
	\index{Stata!Do-files}%
  \index{Replication|seealso{Stata!Do-files}}%
%
% command execution
% comments
% log
%

\newthought{When you run Stata commands} to explore a dataset, you are programming `on the fly', typing commands directly into the Stata command line to get quick results. But when you want to structure a complete analysis of your data, you need to record your commands to a script, which is a plain text file containing Stata code. An important part of your work in this course will therefore be to code your analysis into a script.%

Coding your analysis is meant to provide yourself as well as others with the means to replicate your analysis. Maintaining a record of your operations—and commenting them inside as well as outside the code—will not only ensure that others can read and reproduce your work, but also that \emph{you} will be able to remember, in a few months or so, what you precisely ended up doing on your project.%

Making your research fully replicable is not an option in this course: you are required to provide the code for your analysis. Replication also means that you should \hlred{\textbf{keep your original datasets unmodified}}: you will not need to save your modifications because you will instead save the instructions that allow any Stata user with the original data source to prepare it and analyze it as you have.%

The logic of replicability (or reproducibility) outlined here 

\newthought{Stata scripts} end with the \ext{.do} extension and are called \textbf{do-files}.%
	 \footnote{Stata do-files will usually show as plain text in your Internet browser; if the browser adds a \ext{.txt} extension to a do-file when downloading it, make sure that you rename the file to \ext{.do} for Stata to recognize it as a do-file.} %

It will be one of your main missions throughout the course to learn how to structure and to write such a document. You will be given plenty of examples through a new course do-file every week, and the short vignette that follows will show you some essential aspects of working in Stata from a do-file.%

\begin{marginfigure}
	\includegraphics[width=.75\textwidth]{stata-do-icon}
	\caption{Stata~12 do-file icon.}
	\label{fig:stata-do}
\end{marginfigure}

%
%
\paragraph{Create an empty do-file}

Type \cmd{doedit} in the Command window to open a blank Stata do-file, which is your first piece of draft Stata code. Let's write a few lines into it:%

\begin{docspec}
  * François Briatte, first do-file\\[1em]%
	%
  * Load UN data.\\%
	sysuse lifeexp, clear\\[1em]%
	%
	* Describe the data.\\%
	d\\[1em]%
	%
	* Summarize life expectancy\\%
	su lexp\\[1em]%
	%
	* Scatterplot.\\%
  sc lexp safewater\\[1em]%
	%
	* ttyl\\
\end{docspec}

Here are step-by-step instructions to the code:

\begin{enumerate}
	\index{Stata!Comments (code)}%
	\item First type in your name, preceded by an asterisk and a space. You will notice that the line will turn green, to indicate that it is a \textbf{comment}. Comments are for human readers of the code and are not evaluated by Stata, just passively printed to the Results window.%
	
	\item Now add the \cmd{sysuse} command that asks Stata to load a demo dataset in memory, removing any previous unsaved data. If you have run the previous examples, the same dataset will be loaded again.%
	
	\item Finally, add the \cmd[d]{describe} command to list the variables, the \cmd[su]{summarize} command to inspect the life expectancy variable, and the \cmd[sc]{scatter} command to produce a plot of the relationship between life expectancy and access to safe water.%
\end{enumerate}

Stata shows the code of your do-file in monospaced font with line numbers (so that you can easily navigate long documents) and colored syntax (so that you can distinguish syntax elements). The do-file editor should look like Figure~\ref{fig:hello-world-draft} on my system, which also highlights the currently selected line:%

\begin{figure}%
	\includegraphics[width=\textwidth]{hello-world-draft}%
	\caption{A draft do-file using the \texttt{lifeexp} dataset.}%
	\label{fig:hello-world-draft}%
\end{figure}

Every Stata command requires to be typed on a separate line. You can skip lines and add whitespace in your code to improve its readability.%

\paragraph{Opening and saving do-files}

\newthought{Stata code is saved as plain text} into \ext{.do} files. Stata can open as many do-files as you need in a tabbed window environment, called the Stata do-file editor. You will have to work with course do-files and to write your own for your research project.%

Once you are done copying and editing the demo code from above, use \kbd{Cmd-S} (Mac) or \kbd{Ctrl-S} (Win) to save the do-file as \filename{draft.do} in your \code folder, where you should keep all your do-files within the \SRQM folder.%

Close your first draft and return to the Stata Command window. If you have saved your draft as \texttt{draft.do} into the \code folder, as requested, then Stata will be able to open it with the following command:%

\begin{docspec}
	doedit code/draft
\end{docspec}

The \cmd{doedit} command opens do-files, and the \texttt{code/draft} argument is the short form of \texttt{"(working directory)/code/draft.do"}, a file path that should lead Stata to open the do-file that you just saved.%

Note that \hlred{\textbf{opening a do-file by double-clicking it is not recommended}, because Stata will quietly change the working directory to the location of the do-file.} The same problem arises when you open datasets by double-clicking them. In both cases, you will have to set the working directory back to being the \SRQM folder.%

Similarly, \hlred{\textbf{be careful when you download a do-file}, because your browser might add a \ext{.txt} extension to it}, in which case you will have to to rename the file by turning its file extension back to just \ext{.do} to open it correctly in Stata. Using the ``Save As'' contextual menu option of your browser can fix the issue, as shown in Figure~\ref{fig:save-as} with Google Chrome on \OSX:%

\begin{figure}
	\includegraphics[scale=.5]{images/macosx-save-as.png}
	\caption{The ``Save Link As…'' option in Google Chrome for \OSX.}
	\label{fig:save-as}
\end{figure}

\newthought{Having reopened your do-file}, click anywhere on line 2 of your draft do-file and then press \kbd{Cmd-L} (Mac) or \kbd{Ctrl-L} (Win) to select it in full. Then, press \kbd{Shift+DownArrow} or \kbd{Shift+UpArrow} to select the line below or above it.%

Finally, press \kbd{Cmd-A} (Mac) or \kbd{Ctrl-A} (Win) to select all three lines. These keyboard shortcuts show you how to quickly select one or more lines from your code: make sure to memorize them, as they will come in very handy.%

\paragraph{Running do-files}

To execute a do-file, press \kbd{Cmd-Shift-D} (Mac) or \kbd{Ctrl-D} (Win). The first line is a comment, as indicated by the asterisk at the beginning of the line, so executing it will not accomplish anything: Stata will just print it to the Results window.%

The second and third lines of your do-file will be printed to the Results window with their respective results. The \cmd{cd} command should print the path to your \SRQM folder, and the \cmd{di} command will print a `hello' message, as shown in Figure~\ref{fig:hello-world-result}.%

\begin{figure}%
  \includegraphics[width=\textwidth]{hello-world-result}

  \caption{`Hello World' in Stata.}
  \label{fig:hello-world-result}
\end{figure}

\hlred{\textbf{Important:} check your commands for typos and syntax errors.} If your code fails to execute, Stata will send some red ink to the Results window. Check the code against the correct syntax, looking for forgotten (or extra) letters or spaces.%

\hlred{\textbf{Important:} be careful with copy-pasting to the Command window.} Copy-pasted commands that contain line breaks (\cmd{///}), for example, will not work properly. This point is covered again in the first course do-file.%

\hlred{\textbf{Tip:}} start using keyboard shortcuts as early as possible. For example, during class, you will often need to switch from a window to another one. This can be done from the keyboard: use \kbd{Cmd-`} (Mac) or \kbd{Alt-Tab} (Win). Have a look at the `Window' menu for other shortcuts in: for example, on \OSX, \kbd{Cmd-1} switches to the Command window.%

Homework for this course has three different components, the first of which is the replication do-file. You will be distributed one do-file every week. We will start replicating the do-file in class, and you will be asked to finish replicating it at home.%

\newthought{Your weekly mission} is to replicate each weekly do-file by reading through \filename{week1.do} to \filename{week12.do}. If you have correctly run through the setup described in Section~\ref{sec:course-setup}, the following command will open the do-file for Week~1:%

\begin{docspec}
  doedit code/week1.do
\end{docspec}

Read through the do-file and execute (run) its commands sequentially, after reading the comments that precede each block of code. This practice (replicating the course do-files) will show you how to code various things in Stata.%

\newthought{There is one do-file to replicate per week of class.} The do-file for the first week covers basic Stata settings and data exploration commands. Each section of this guide opens on a short introduction that will indicate which do-file it corresponds to.%

%%% -- TODO: rewrite
\paragraph{Logging your work}

The log is a text file that, once open with log using, will save every single command you enter in Stata as well as its results. Systematically logging your work is good prac-tice, even when you are just trying out a few things. Logs can be closed with the log close command followed by the name of your log if it has one:

Comments will also be saved to the log file, which is particularly useful when you have to read through your work again or share it with someone. In the example above, all comments, commands and results were saved to the log file.


Logs are useful to save every operation and result from a practice session. If you need someone else to replicate your work, however, you just need to share the commands you entered, along with the comments that you wrote to document your analysis. Files that contain commands and comments are called do-files.
Writing do-files is a crucial aspect of this course. Absent of a do-file, your work will be mostly incomprehensible, or at least impossible to reproduce, to others. Your do-file should include your comments, and it should run smoothly, without returning any er-rors. You will discover that these steps require a lot of work, so start to program early. To open a new do-file, use either the doedit command or the ‘File > New Do-file’ menu (keyboard shortcut: Ctrl-N on Windows, Cmmd-N on Macintosh).
You should take inspiration from the do-files produced for the course to write up your own do-file for your research project. All our do-files are available from the course web-site. This course requires only basic programming skills, as illustrated by the do-files that we run during our practice sessions. More sophisticated examples can be found online.
To execute (or ‘run’) a do-file, open it, select any number of lines, and press Ctrl-D in Stata for Windows or Cmmd-Shift-D in Stata for Macintosh. You can also use either the GUI icons on the top-right of the Do-file Editor window, or use the do or run com-mands. Use the Ctrl-L (Windows) or Cmmd-L (Macintosh) keyboard shortcuts to select the entire current line in order to run it.
Get some practice with do-files as soon as possible, since your coursework will include replicating one do-file a week. Replicating is nothing more than reading through the comments of a do-file, while running all its commands sequentially.

% % %

\paragraph{Exiting}

When you are done with your work, just quit Stata like you would quit any other program. Stata will ask you whether you want to save your changes: \textbf{say no}.%

In this course, you will \hlred{never save any changes to the original datasets}: keep the teaching material intact at all times, so that it can be used throughout the semester.%

More generally, because you will be coding every modification that you make to the course datasets, your analysis will be fully replicable from the original datasets.%

\newthought{The nerdy way to quit Stata} is to quit from the command line. A first optional step is to manually close all opened logs:%

\begin{docspec}
	* Close all open logs.\\%
	log close \_all
\end{docspec}

The \cmd{exit} command with the \opt{clear}{exit} option then erases any data in memory and quits:%

\begin{docspec}
	* Enjoy your day.\\%
	exit, clear
\end{docspec}
